%%% LaTeX Template: Two column article
%%%
%%% Source: http://www.howtotex.com/
%%% Feel free to distribute this template, but please keep to referal to http://www.howtotex.com/ here.
%%% Date: February 2011

%%% Preamble
\documentclass[	DIV=calc,%
							paper=a4,%
							fontsize=11pt,%
							twocolumn, draft]{scrartcl}	 					% KOMA-article class

%S\usepackage{lipsum}		



\usepackage[english]{babel}										% English language/hyphenation
%\usepackage[protrusion=true,expansion=true]{microtype}				% Better typography
%\usepackage{amsmath,amsfonts,amsthm}					% Math packages
%\usepackage[pdftex]{graphicx}									% Enable pdflatex
\usepackage[svgnames]{xcolor}									% Enabling colors by their 'svgnames'
%\usepackage[hang, small,labelfont=bf,up,textfont=it,up]{caption}	% Custom captions under/above floats
%\usepackage{epstopdf}												% Converts .eps to .pdf
%\usepackage{subfig}													% Subfigures
%\usepackage{booktabs}												% Nicer tables
\usepackage{fix-cm}													% Custom fontsizes

\usepackage{draftwatermark}
\SetWatermarkText{Draft}
\SetWatermarkScale{10}% Package to create dummy text

%%% Custom sectioning (sectsty package)
%\usepackage{sectsty}													% Custom sectioning (see below)
%\allsectionsfont{%															% Change font of al section commands
%	\usefont{OT1}{phv}{b}{n}%										% bch-b-n: CharterBT-Bold font
%	}

%\sectionfont{%																% Change font of \section command
%	\usefont{OT1}{phv}{b}{n}%										% bch-b-n: CharterBT-Bold font
%	}



%%% Headers and footers
%\usepackage{fancyhdr}												% Needed to define custom headers/footers
%	\pagestyle{fancy}														% Enabling the custom headers/footers
%\usepackage{lastpage}	

% Header (empty)
%\lhead{}
%\chead{}
%\rhead{}
% Footer (you may change this to your own needs)
%\lfoot{\footnotesize \texttt{HowToTeX.com} \textbullet ~Two column article template}
%\cfoot{}
%\rfoot{\footnotesize page \thepage\ of \pageref{LastPage}}	% "Page 1 of 2"
%\renewcommand{\headrulewidth}{0.0pt}
%\renewcommand{\footrulewidth}{0.4pt}



%%% Creating an initial of the very first character of the content
\usepackage{lettrine}
\newcommand{\initial}[1]{%
     \lettrine[lines=3,lhang=0.3,nindent=0em]{
     				\color{DarkGoldenrod}
     				{\textsf{#1}}}{}}



%%% Title, author and date metadata
\usepackage{titling}															% For custom titles

\newcommand{\HorRule}{\color{DarkGoldenrod}%			% Creating a horizontal rule
									  	\rule{\linewidth}{1pt}%
										}
%%begin novalidate
\pretitle{\vspace{-30pt} \begin{flushleft} \HorRule 
				\fontsize{50}{50} \usefont{OT1}{phv}{b}{n} \color{DarkRed} \selectfont 
				}
\title{TOWARDS AN ETHICS-BY-DESIGN APPROACH IN DATA EXPERIMENTATION PROJECTS}					% Title of your article goes here
\posttitle{\par\end{flushleft}\vskip 0.5em}

%\preauthor{\begin{flushleft}
%					\large \lineskip 0.5em \usefont{OT1}{phv}{b}{sl} \color{DarkRed}}
%\author{Till Riedel, }											% Author name goes here
%\postauthor{\footnotesize \usefont{OT1}{phv}{m}{sl} \color{Black} 
%					Karlsruhe Instititute of Technology (KIT)							% Institution of author
%					\par\end{flushleft}\HorRule}
%%end novalidate
\date{}																				% No date



%%% Begin document
\begin{document}
\maketitle
%\thispagestyle{fancy} 			% Enabling the custom headers/footers for the first page 
% The first character should be within \initial{}
\initial{H}\textbf{ere is some sample text to show the initial in the introductory paragraph of this template article.}

Within the Horizon 2020 project EUHubs4Data \cite{EUH4D} data experimentation projects
with small and medium enterprises (SMEs) were
set up to show the case and accelerate data innovation. We ourselves
were funded by the EUHubs4Data project which consisted of 42
``experiments''. Experimentation in this sense
meant that a controlled environment between a group of partners was created to form a federation of data-driven innovation hubs
that set up a cross-sectoral and cross-border data space. 
This space emulated a diverse ecosystem, as we might see in future data spaces currently established across the European Union as recently defined by the Data Act \cite{} which became applicable law on January 11 2024, just as the project finalizes.

One important aspect of such experimentation is to validate concepts and assumptions, but also to discover unknown and sometimes adverse effects.
Such adverse effects around the innovative use of data often relate to ethical issues. The Human Rights Council of the UN already affirmed in its resolution on the promotion and protection of human rights on the internet, in July 2012, “that the same rights that people have offline must also be protected online, in particular freedom of expression”. Since then things have developed rapidly and particularly the free flow of data was enabled by the internet. Recently, particularly the use of artificial intelligence, which also in our project played the most prominent role in data innovation, became a major concern towards human rights. The AI Act \cite{AIAct}, which is expected to become European legislation within 2024 even mandated human rights impact assessments for using AI models in certain "high risk" applications. Many of the impacts novel use of data to derive knowledge or autonomous agents is unknown.

The experiments, we will reflect on in this whitepaper, were led by innovative SMEs that were
independently selected in the so-called `open calls'. They were supported by a number predefined ecosystem members (so-called data innovation hubs or i-Spaces
in our case) that were directly funded to provide an infrastructure for
experimentation. Typically, this setting emulates a market situation
using a public offering. However, being carried out within a research
and innovation project, the situation differs because SMEs can use
public funding to cover both covering cost of the offered and their own
work.

In such a setting the ethical impact is a big unknown at the start. So we set out to monitor the experiments and the overall project \cite{D3.1}. In the end one of the core distinguishing aspects of the upcoming European economy is supposed to be data and AI ethics \cite{some digital strategy}.  After three years and having
monitored 42 very different experiments all around data, it is time to review and reflect on our learnings, on the state of our ecosystem (which we try in the first section). One of the findings that we will present is that, after five years of General Data Protection Regulation, SMEs still have difficulties achieving basic levels of compliance when dealing with large amounts of data. From what we have seen throughout the project, we see the importance of practical frameworks and support structures, that will become even more important when implementing trustworthy AI. This is why wee in the second part of this whitepaper will argue for the need of further experimentation as a form of exercise to actually live by the standards we set ourselves. Based on our own experiences, we see the need to share our tools that practically enable ethics, 
but also basic legal and contractual compliance, as SMEs are increasingly overwhelmed by demands imposed on them.


\section{42 experiments under the ethical looking glas}\label{}

After reviewing a set of ten first projects from we already published a first report on our findings as a public report. In this report \cite{DXX}, 
we already summarized many of the challenges. In this whitepaper and 32 experiments later, we would like to briefly summarize the areas that appeared critical during our ethics monitoring.
As a methodology, we reviewed all 42 project at start, midterm and after finalization. We did this as a group of technical experts with a lot of practical experience in the area of data innovation that was advised by a legal expert. As we started, we had little tools and were expecting to focus on rather theoretical edge cases of unintended use of data and AI models. However, as we discovered, and this is the probably the major finding of our extensive experimentation, the majority of applications are touching real hazards around data. In the majority of those cases, the biggest problem is that it remained unclear from the initial documentation if real ethical risks can be foreseen. As a dynamic of any funding application, one can easily imagine the positive impacts and the foreseen scaling of data processing. What can be said in general: the negative impact and adversarial effects foreseen were clearly disproportionate for most projects at the start of experimentation. We believe that ethics by design and default first of all requires awareness of such hazards, which we tried to provide as feedback to the experiments.

\subsection{Data Protection of Personal Data}\label{}
Examples, examples, examples

\subsection{Potential adversarial effects of the use of AI}\label{}
Examples, examples, examples


\subsection{Cybersecurity, Fair competition and other issues}\label{}



\section{Learning ethics for real}\label{learning-ethics-for-real}

Ideally, we would be able to take our learnings straight from the lab to
reality. As already touched upon, our project has been set up to emulate
a market-oriented data economy; however, if you particularly look at
ethics, we see  differences.

One of the core pillars of ethics is understanding and accepting
responsibility. In a cascade-funded setting this is not easy. As money
is passed from the European commission to coordinator onto subcontracted
SMEs, a legal and contractional regime is established that sets the
playing field for many things that follow. Ethical actions require the
choice to do things in the right way. This goes both for the SME
performing the data experiment (which had to work with the services and
data provided by the framework project to funding) and the data
innovation hubs (which in turn had to work for the projects that were
selected by external reviewers). In such a setting, there is not that
much room for deciding for shared values (if you do not consider public
money as shared value in itself).

We think it is important to acknowledge that precompetitive and funded
environments have a dynamic of their own. They are important for
progress towards a true European data economy; thus, we will focus our
analysis mostly on this setting. We do this in the hope that other
project that have similar mechanics can learn from us. Constant progress
also in developing ethical frameworks can also have a positive impact on
the market.

\subsection{Competition of values}\label{competition-of-values}

\textbf{Funded projects must be governed by choices and finding partners
that share values. Also, in a funded setting, we need positive
competition around trustworthy and responsible data innovations.}

To be clear: we have seen external reviewers choose, let us say,
challenging projects. Often generating a major positive impact has the
risk that if done wrongly, it may a trigger also negative impact. In our
public report on the findings from the first open call, we have listed
many different ethical challenges that we encountered. We did not feel
prepared for all these challenges (from conducting medical trials to
dealing with financial transaction data) and had to rely on the
competence of the SMEs conducting the experiments. In many cases that
worked out well, however, in many cases elimination of ethical risk was
not an option because the SME was funded because it promised an output,
and the data innovation hubs were funded to support the SMEs. Failure on
both sides was not really an option in order to bring the overall
project to an end after the fair and independent selection of the open
call was finished. We learnt by putting more and more `terms and
conditions'' online. However, a balancing of risk and impact during the
selection process based also on the non-formalised values of the
infrastructure would have (e.g. by sometimes selecting the second-most
impactful experiment, if it has better compatibility with the
self-determined competences and values of the infrastructure providers).

\subsection{Compliance as a baseline}\label{compliance-as-a-baseline}

\textbf{We cannot argue that legal compliance is a given, and ethics
should only go beyond this.}

Most of the risks detected by the ethics monitoring group of our project
centred on GDPR compliance. With the GDPR in place for more than 10
years, it would expect that it should be at the core of all data
processing that involves data related to human data subjects. The
reality is different. Even a common understanding of terminology is
complex. If you look at anonymity, there are on one hand certainly
difficult edge cases that have required more recent court rulings. An
example might be the definition of personal data as clarified in the
famous Patric Breyer case (ECLI:EU:C:2016:779). However, the reality is
that the difference between anonymity and pseudonymity is often
understood even in simple cases. Many big companies have invested in
compliance, also due to clear requirements and considerable possible
fines. The SME space, but also the research system, is from our
experience only very slowly catching up given exemptions and lack of
enforcement in this domain. This fact makes it very hard to establish
ethics monitoring, saying repeatedly to people `I'm not a lawyer, but I
think what you're doing might not be legal this way.' The argument that
data coming from the EU is more `ethical' due to a clear legal regime
does not hold. We have seen in our experiments multiple public data sets
which were published by EU projects for which we could not clearly
determine a legal basis.

\subsection{Setting values}\label{setting-values}

As said above, over the course of our project, the terms and conditions
of the open calls evolved. Also, some service providers set conditions
on what they would support and what not. Often those terms were only
there to have a better lever at enforcing compliance, particularly with
the contractual requirements of the grant (namely the ethical impact
assessment of the project and evidence collection required by the funding
program).

\ldots{}


\end{document}