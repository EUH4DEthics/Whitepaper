\section{Conclusion and outlook}
The vision of the EUHubs4Data project was to establish a sustainable federation of data innovation hubs that span a cross-sectorial data space. While bringing such federated data spaces is still an ongoing effort. Experimentation under different project regimes will continue to be an important aspect of establishing such innovation ecosystems. We believe that such experimentation ecosystems should also be used to evolve ethical frameworks. For the monitoring work that we conducted within the project, which has ended now, we made important findings on both regulatory compliance and risk-based ethics monitoring. To say it bluntly: SMEs have not managed to keep up with the expectations of a fully trustworthy European data economy. Just following and understanding the underlying concept and the underlying regulations can be overwhelming. Even more,  SMEs often seem successful in data innovation by solely focusing their resources on the potential positive impact there. Exercising the ethical muscle should not lead to effects of "moral fatigue" which should come with rewards.

By providing and evolving practical tools and material directed at experimentation facilities, we hope to be able to contribute in scaling up a European data economy. By giving SMEs feedback and helping them evolve their innovations, we also received positive reinforcement on ethical considerations. Without such incentives and feedback inside funding regimes, we observed little incentive to incorporate "ethics by design" before entering the market for the "typical SME". On the other hand, we believe that by establishing a monitoring regime, we helped to derive more market-ready products that will also better meet legislative requirements and fulfill market needs around trustworthiness. We therefore hope to see more funding regimes adopt ethics frameworks that align the "exercise" of European values.